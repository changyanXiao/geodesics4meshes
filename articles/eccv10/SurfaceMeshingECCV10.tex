\documentclass[runningheads]{style/llncs}

\usepackage{graphicx}
\usepackage{amsmath,amssymb} % define this before the line numbering.
\usepackage{lineno}
\usepackage{color}

\usepackage{style/gpeyre}
\usepackage{style/geodesic}
\usepackage{url}


\usepackage{epsfig}
\usepackage{epstopdf}

\DeclareGraphicsExtensions{.png,.pdf,.eps,.jpg} 
\graphicspath{{./images/}}


\newcommand{\ecc}{\text{E}}
% distortion of mapping
\newcommand{\distF}{\de}
% distance between shapes
\newcommand{\distS}{\De}

\def\ie{{\em {i.e.}~}}
\def\eg{{\em {e.g.}~}}
\def\etc{{\em {etc}~}}
\def\etal{{\em {et al.}~}}

\usepackage{subfigure}
% \usepackage[english]{babel}
% \usepackage{latexsym}
% \usepackage[T1]{fontenc}
% \usepackage{hyperref}

\begin{document}
\renewcommand\thelinenumber{\color[rgb]{0.2,0.5,0.8}\normalfont\sffamily\scriptsize\arabic{linenumber}\color[rgb]{0,0,0}}
\renewcommand\makeLineNumber {\hss\thelinenumber\ \hspace{6mm} \rlap{\hskip\textwidth\ \hspace{6.5mm}\thelinenumber}} 
\linenumbers
\pagestyle{headings}
\mainmatter
\def\ECCV10SubNumber{***}  % Insert your submission number here

\title{Anisotropic Geodesic Remeshing} % Replace with your title

\titlerunning{Anisotropic Geodesic Remeshing}

\authorrunning{Paper ID \ECCV10SubNumber}

\author{Anonymous ECCV submission}
\institute{Paper ID \ECCV10SubNumber}


\maketitle

\begin{abstract}
	This paper presents a new method for the anisotropic remeshing of 3D surfaces.
	Starting from an densely sampled 3D mesh, the method build a curvature dependent 
	tensor field, which defines a Riemannian metric that drives the seeding of triangles.
	The core of the algorithm is a interleaving of geodesic farthest point seeding and 
	a geodesic Lloyd relaxation. This allows for a high quality placement of points according to the metric, 
	and the estimation of an anisotropic geodesic Delaunay triangulation.
\end{abstract}

%%%%%%%%%%%%%%%%%%%%%%%%%%%%%%%%%%%%%%%%%%%%%%%%%%%%%%%
%%%%%%%%%%%%%%%%%%%%%%%%%%%%%%%%%%%%%%%%%%%%%%%%%%%%%%%
%%%%%%%%%%%%%%%%%%%%%%%%%%%%%%%%%%%%%%%%%%%%%%%%%%%%%%%
\section{Introduction}

This paper tackles the difficult problem of meshing a surface with anisotropic triangles. 
Both the density and anisotropic of the triangles should be adapted to capture the complex regularity of 
a 3D surface. 


%%%%%%%%%%%%%%%%%%%%%%%%%%%%%%%%%%%%%%%%%%%%%%%%%%%%%%%
\subsection{Previous Works}

%%
\paragraph{Isotropic meshing.}

%%
\paragraph{Anisotropic planar domain meshing.}

Optimal shape of triangles to approximation a smooth function.

%%
\paragraph{Anisotropic surface meshing.}

Optimal shape to approximate a smooth surface according to the Hausdorf metric.

%%%%%%%%%%%%%%%%%%%%%%%%%%%%%%%%%%%%%%%%%%%%%%%%%%%%%%%
\subsection{Contributions}



%%%%%%%%%%%%%%%%%%%%%%%%%%%%%%%%%%%%%%%%%%%%%%%%%%%%%%%
%%%%%%%%%%%%%%%%%%%%%%%%%%%%%%%%%%%%%%%%%%%%%%%%%%%%%%%
%%%%%%%%%%%%%%%%%%%%%%%%%%%%%%%%%%%%%%%%%%%%%%%%%%%%%%%
\section{Geodesic Computations on Surfaces}

%%%%%%%%%%%%%%%%%%%%%%%%%%%%%%%%%%%%%%%%%%%%%%%%%%%%%%%
\subsection{Surfaces as Riemannian Manifolds}

We consider here a smooth
surface $\surf \subset \RR^3$ embedded into the 3D Euclidean space.

To define the length of a curve drawn on the surface, a tensor $\tensor{x} \in \RR^{3 \times 3}$ is attached to 
each point $x \in \surf$. This tensor allows one to compute the length of a vector according to 
\eq{
	\normT{u}{\tensor{x}}^2 =  \dotpT{u}{u}{\tensor{x}}
	\qwhereq
	\dotpT{u}{v}{\tensor{x}} = 
	\dotp{\tensor{x} u}{v} = \sum_{1 \leq i,j \leq d} (\tensor{x})_{i,j} u_i v_j.
}
Since this tensor is intended to measure length of tangent to curves traced on the surface, it is assumed to be vanishing on the normal direction $n_x$ at $x$, which means $\normT{u}{\tensor{x}} = 0$.

The length of a curve $\ga : [0,1] \rightarrow \surf$ traced on the surface is then defined as
\eql{\label{eq-length-surf-embedding} 
	\length(\ga) = \int_0^1
  		\normT{\ga'(t)}{\tensor{\ga(t)}} \d t, 		} 
where $\ga'(t) \in \tangentPlane_{\ga(t)}�\subset \RR^3$ is the derivative vector, 
that lies in the embedding space $\RR^3$,
and is in fact a vector belonging to the 2D tangent plane $\tangentPlane_{\ga(t)}$ to the surface at
$\ga(t)$.

The geodesic distance between two points $\xstart,\xend \in \surf$ is the length of the shortest curve joining the two points 
\eql{\label{eq-dfn-geodesic-dist}
	d(\xstart,\xend) = \umin{\ga \in \joining(\xstart,\xend)} \length(\ga)
}
where we use the following set of piecewise smooth curves 
\eq{
	\joining(\xstart,\xend) = \enscond{\ga : [0,1] \rightarrow \surf}{\ga(0)=\xstart, \ga(1)=\xend}.
}
A curve $\ga^\star$ satisfying $d(\xstart,\xend) = \length(\ga^\star)$ is called a shortest path, sometimes also referred to as a (globally minimizing) geodesic.


\myfigure{ 
\tabdeux{
\image{0.4}{metric/metric-surface}&
\image{0.4}{metric/geodesic-surface}\\
Surface $\surf$ & Geodesic distance and paths.
}
}{
Example of geodesic curves on a 3D surface. The color display the function $x \mapsto d(\xstart,\xend)$ 
and the curves are geodesics between a single $\xstart$ and several $\xend$.% 
}{fig-geodesics-surface} 

Figure \ref{fig-geodesics-surface} shows an example of a 3D surface, together with the geodesic distance to $\xstart$ and several geodesic curves.



%%%%%%%%%%%%%%%%%%%%%%%%%%%%%%%%%%%%%%%%%%%%%%%%%%%%%%%
\subsection{Eikonal Equation and Geodesics}


%%%%%%%%%%%%%%%%%%%%%%%%%%%%%%%%%%%%%%%%%%%%%%%%%%%%%%%
\subsection{Numerical Computation of Geodesics on Surfaces}

%%%%%%%%%%%%%%%%%%%%%%%%%%%%%%%%%%%%%%%%%%%%%%%%%%%%%%%
%%%%%%%%%%%%%%%%%%%%%%%%%%%%%%%%%%%%%%%%%%%%%%%%%%%%%%%
%%%%%%%%%%%%%%%%%%%%%%%%%%%%%%%%%%%%%%%%%%%%%%%%%%%%%%%
\section{Geodesic Farthest Point Sampling}


%%%%%%%%%%%%%%%%%%%%%%%%%%%%%%%%%%%%%%%%%%%%%%%%%%%%%%%
%%%%%%%%%%%%%%%%%%%%%%%%%%%%%%%%%%%%%%%%%%%%%%%%%%%%%%%
%%%%%%%%%%%%%%%%%%%%%%%%%%%%%%%%%%%%%%%%%%%%%%%%%%%%%%%
\section{Geodesic Lloyd Relaxation}




%%%%%%%%%%%%%%%%%%%%%%%%%%%%%%%%%%%%%%%%%%%%%%%%%%%%%%%
%%%%%%%%%%%%%%%%%%%%%%%%%%%%%%%%%%%%%%%%%%%%%%%%%%%%%%%
%%%%%%%%%%%%%%%%%%%%%%%%%%%%%%%%%%%%%%%%%%%%%%%%%%%%%%%
\section*{Conclusion}




%%%%%%%%%%%%%%%%%%%%%%%%%%%%%%%%%%%%%%%%%%%%%%%%%%%%%%%
%%%%%%%%%%%%%%%%%%%%%%%%%%%%%%%%%%%%%%%%%%%%%%%%%%%%%%%
%%%%%%%%%%%%%%%%%%%%%%%%%%%%%%%%%%%%%%%%%%%%%%%%%%%%%%%
\bibliographystyle{style/splncs}


\bibliography{bib/bib-abbrev,bib/bib-books,bib/bib-fast-marching,bib/bib-maths,bib/bib-meshes,bib/bib-sampling}



\end{document}
